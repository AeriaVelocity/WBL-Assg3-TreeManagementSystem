\documentclass[12pt]{report}
\usepackage{librebaskerville}
\usepackage{datetime2}
\usepackage{parskip}
\usepackage{hyperref}
\hypersetup{
    colorlinks=true,
    linkcolor=black,
}
\raggedright{}
\linespread{1.6}
\usepackage{geometry}
\geometry{margin=1in}
\title{WBL Tree Management System --- Final Written Report}
\author{Arsalan Kazmi}
\date{2024--02--16 --- \today}

\begin{document}

\maketitle

\tableofcontents

\section{Abstract}

% Do this when all is done

\section{Introduction}

\subsection{This Report}

This written report will discuss and detail the Tree Management System,
developed for Highfield Park Trust, reflecting on the project as a whole and
detailing each aspect of it, including its project methodology, its design, how
it was developed and tested, the end results, recommendations for future
implementation and any challenges that occurred along the way.

\subsection{Organisation}

Highfield Park Trust is an independent charity formed in 1996, which manage,
preserve, protect, develop and improve the features of Highfield Park, which
is situated in St Albans. (Highfield Park Trust, 2023)

\subsection{Project Background}

The Tree Management System aims to provide the Trust with a reliable computerised
system which can be utilised for easy and robust management of trees, allowing
authorised individuals to report any issues concerning a specific tree to a
central server inside the Park, and allowing members of the public to make
comments if they notice something an administrator may have overlooked.

\subsection{Contents Overview}

\textbf{Project Methodology} --- Goes over the project methodology, research
methods, objectives and requirements of the project, as well as the resources
used throughout the project.

\textbf{Design} --- Goes over the design of the project, including the
methods used to develop it and the environment in which it was developed.

\textbf{Development and Testing} --- Goes over the development and testing
phases of the project, detailing how development was done and the test plan
that was constructed.

\textbf{Results and Discussion}

\textbf{Recommendations}

\textbf{Challenges and Reflection}

\textbf{Conclusion}

\section{Project Methodology}

\subsection{Project Methodologies and Research Methods}

The TMS project used the \textbf{Agile} methodology, with the \textbf{Scrum}
method. The Agile methodology breaks a project into separate stages, enabling
continuous collaboration and improvement, following a cycle of planning,
executing, and evaluating (Atlassian, n.d.), and Scrum is a framework
which utilises adaptive solutions for complex problems, divided into
``sprints''. (Schwaber \& Sutherland, 2020)

This allowed the project to be divided into 5 phases, which were much more
manageable and efficient than simply doing all the work in one stage, or
dividing the work into too many stages.

\subsection{Project Objectives}

\begin{itemize}
    \item \textbf{Research} --- data collection technologies for tree tracking,
    data storage methods, for the purpose of an efficient tree management
    system, that takes into account other people's research.
    \item \textbf{Designing} --- create a front-end and back-end design for the
    web application, to get a clear graphical representation of elements of the
    application, using wireframes, mock-ups and database schemas. 
    \item \textbf{Development} --- Implement a web application, using web
    technologies, such as React.
    \item \textbf{Testing} --- Make sure every iteration and component is
    working, and complete the final usability and integration testing using a
    comprehensive test plan.
    \item \textbf{Report authoring} --- Collate all the documents from the
    previous stages to complete a final project report.
\end{itemize}

\subsection{Requirement Analysis}

\subsubsection{Core}

\begin{itemize}
    \item How would the user be able to collect data? i.e.\ forms, logging in,
    collecting data, outputting data, converting data into a report
    \item Ensure that the data collection is kept entirely secure.
    \item Collect analytics from the use of the web app, but make it an opt-in.
\end{itemize}

\subsection{Research Methods}

For the TMS project, the following primary research methods were used:

\begin{itemize}
    \item \textbf{Interview} --- Interviews conducted with people who were
    in the field of managing and caring for trees, forests and other nature
    reserves.
    \item \textbf{Observation} --- Observations of people who spend time in
    nature reserves, noting their behaviour, reasons for being there and
    whatever situations they may be in.
\end{itemize}

Additionally, the following secondary research methods were used:

\begin{itemize}
    \item \textbf{Metadata collection} --- Collecting data on the tree database
    itself, for example, what people may comment about certain trees, how
    quickly actions by authorised individuals are done, etc.
    \item \textbf{Web Search} --- Searching for information online for how
    people have managed natural environments in the past, including systems
    similar to the proposed TMS.\
\end{itemize}

\section{Design}

\subsection{Methods Used}



\subsection{Development Environment}

\subsection{Uncertainties}

\subsection{Software Usage}

\subsection{Intermediate Results}

\section{Development and Testing}

\section{Results and Discussion}

\section{Recommendations}

\section{Challenges and Reflection}

\section{Conclusion}

\section{References}
% BibTeX is too much of a headache

\begin{itemize}
    \item Atlassian (n.d.) \textit{What is agile?}, Atlassian. Available at:
      https://www.atlassian.com/agile (Accessed: 01 March 2024).
    \item Highfield Park Trust (2023) \textit{About Us | Highfield Park Trust},
      Highfield Park Trust. Available at:
      https://www.highfieldparktrust.co.uk/about-us/ (Accessed: 21 February
      2024)
    \item Schwaber, K. and Sutherland, J. (2020) \textit{The 2020 Scrum Guide,
      Scrum Guide | Scrum Guides.} Available at:
      https://scrumguides.org/scrum-guide.html (Accessed: 01 March 2024).
\end{itemize}

\section{Appendix} % >:C this is LaTeX erasure

\end{document}
